% r.9 introduction
\cleardoublepage
\chapter{Mathematics}
\label{ch:math}

\newthought{This chapter} reviews all the basic mathematics required for electrical engineering. 


%%
% Start the main matter (normal chapters)
%\mainmatter

\section{Calculus}
\paragraph{Functions} are the backbone of Calculus and is typically viewed in terms of their graphs of $x$ paired with $f(x)$. It can be viewed more generally as a {\it mechanism} of producing an output $f(x)$ for an {\it given} input $x$. With this mechanical view in mind, certain properties of functions can be defined more naturally. The {\it domain} of a function consists of all possible values of inputs and the {\it range} is all possible values of outputs. For single-variable calculus, the domain and ranges are going to be simple for example the real number line $R=(-\infty ~\infty)$ or certain subsets of that $[a,b], [0, \infty], etc.$.
